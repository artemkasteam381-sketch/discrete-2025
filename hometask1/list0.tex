\documentclass{article}
\usepackage{amsmath, amsthm, amssymb, amsfonts}
\usepackage{thmtools}
\usepackage{graphicx}
\usepackage{indentfirst}
\usepackage{setspace}
\usepackage{geometry}
\usepackage{float}
\usepackage{hyperref}
\usepackage{cancel}
\usepackage[utf8]{inputenc}
\usepackage[russian]{babel}
\usepackage{framed}
\usepackage[dvipsnames]{xcolor}
\usepackage{tcolorbox}
\usepackage[textsize=small, textwidth=3.5cm]{todonotes}
\setlength{\marginparwidth}{1.5cm}

\colorlet{LightGray}{White!90!Periwinkle}
\colorlet{LightOrange}{Orange!15}
\colorlet{LightGreen}{Green!15}

\newcommand{\HRule}[1]{\rule{\linewidth}{#1}}

% Теоремные окружения и стили
\declaretheoremstyle[name=Theorem,]{thmsty}
\declaretheorem[style=thmsty,numberwithin=section]{theorem}
\tcolorboxenvironment{theorem}{colback=LightGray}

\declaretheoremstyle[name=Conjecture,]{thmsty}
\declaretheorem[style=thmsty,numberwithin=section]{conjecture}
\tcolorboxenvironment{conjecture}{colback=LightGray}

\declaretheoremstyle[name=Definition,]{thmsty}
\declaretheorem[style=thmsty,numberwithin=section]{definition}
\tcolorboxenvironment{definition}{colback=LightGray}

% Стиль задач: показываем пометку \NOTE в квадратных скобках и русское имя
\declaretheoremstyle[name=Задача,notebraces={[}{]}]{prosty}
\declaretheorem[style=prosty,numberlike=theorem]{task}

\setstretch{1.2}
\geometry{
    textheight=9in,
    textwidth=5.5in,
    top=1in,
    headheight=12pt,
    headsep=25pt,
    footskip=30pt,
    right=4.5cm
}

\setuptodonotes{size=\small, color=white, bordercolor=white, textcolor=Bittersweet}

\title{Листочек №1 ``Теория множеств''}
\author{Матмех, группы 25.Б82-мм}
\date{Октябрь 2024}

\begin{document}
\maketitle

\section{Задачи}

% 1. Множества — эквивалентные формулировки
\begin{task}[1]
Доказать, что
\begin{enumerate}
    \item[а)] $A \subseteq B \cap C$ тогда и только тогда, когда $A \subseteq B$ и $A \subseteq C$;
    \item[б)] $A \subseteq B \setminus C$ тогда и только тогда, когда $A \subseteq B$ и $A \cap C = \varnothing$.
\end{enumerate}
\end{task}

% 2. Мощности множеств — равенства и включения
\begin{task}[2]
Доказать следующие равенства и включения:
\begin{enumerate}
    \item[а)] $\mathcal{P}(A \cap B) = \mathcal{P}(A) \cap \mathcal{P}(B)$;
    \item[б)] $\mathcal{P}(A \cup B) \supseteq \mathcal{P}(A) \cup \mathcal{P}(B)$;
    \item[в)] $\mathcal{P}(A \setminus B) \subseteq (\mathcal{P}(A) \setminus \mathcal{P}(B)) \cup \{\varnothing\}$.
\end{enumerate}
Привести примеры, когда указанные включения являются строгими.
\end{task}

% 3. Алфавиты и порядки
\begin{task}[3]
Пусть $A = \{a_1, \ldots, a_m\}$ — конечный алфавит, $A^n$ — множество слов длины $n$ в алфавите $A$.
\begin{enumerate}
    \item[(a)] На $A^n$ задано отношение $R_1$: для $v = a_{i_1}\ldots a_{i_n}$ и $w = a_{j_1}\ldots a_{j_n}$ положим $(v,w)\in R_1$ тогда и только тогда, когда $i_k\le j_k$ для всех $k=1,\dots,n$ и $i_k<j_k$ для некоторого $k$. Является ли $R_1$ отношением частичного (линейного) порядка?
    \item[(б)] На $A^*$ задано отношение $R_2$: для $v = a_{i_1}\ldots a_{i_n}$ и $w = a_{j_1}\ldots a_{j_r}$ положим $(v,w)\in R_2$ тогда и только тогда, когда существует $k$ от $1$ до $n$ с $i_\ell=j_\ell$ при $1\le \ell<k$ и $i_k<j_k$, причём первые $n$ символов $w$ совпадают со словом $v$. Является ли $R_2$ отношением частичного (линейного) порядка?
\end{enumerate}
\end{task}

% 4. Функции — область значений и свойства
\begin{task}[2]
Для каждой из функций найти область значений и указать, является ли функция инъективной, сюръективной, биекцией.
\begin{enumerate}
    \item[(а)] $f : \mathbb{R} \to \mathbb{R},\ f(x) = 3x + 1$;
    \item[(б)] $f : \mathbb{R} \to \mathbb{R},\ f(x) = x^2 + 1$;
    \item[(в)] $f : \mathbb{R} \to \mathbb{R},\ f(x) = x^3 - 1$;
    \item[(г)] $f : \mathbb{R} \to \mathbb{R},\ f(x) = e^x$;
    \item[(д)] $f : \mathbb{R} \to \mathbb{R},\ f(x) = \sqrt{3x^2 + 1}$;
    \item[(е)] $f : [-\pi/2, \pi/2] \to \mathbb{R},\ f(x) = \sin x$;
    \item[(ж)] $f : [0, \pi] \to \mathbb{R},\ f(x) = \sin x$;
    \item[(з)] $f : \mathbb{R} \to [-1, 1],\ f(x) = \sin x$;
    \item[(и)] $f : \mathbb{R} \to \mathbb{R},\ f(x) = x^2 \sin x$.
\end{enumerate}
\end{task}

% 5. Композиция функций — логические следствия
\begin{task}[2]
Даны $g : A \to B$ и $f : B \to C$. Рассмотрим композицию $g\circ f : A \to C$, $(g\circ f)(x)=f(g(x))$. Определить, какие утверждения верны:
\begin{enumerate}
    \item[(а)] Если $g$ инъективна, то $g\circ f$ инъективна.
    \item[(б)] Если $f$ и $g$ сюръективны, то $g\circ f$ сюръективна.
    \item[(в)] Если $f$ и $g$ биекции, то $g\circ f$ биекция.
    \item[(г)] Если $g\circ f$ инъективна, то $f$ инъективна.
    \item[(д)] Если $g\circ f$ инъективна, то $g$ инъективна.
    \item[(е)] Если $g\circ f$ сюръективна, то $f$ сюръективна.
\end{enumerate}
\end{task}

% 6. Кафе-мороженое (парадокс ссор)
\begin{task}[3]
Учащиеся одной школы часто собираются группами и ходят в кафе-мороженое. После такого посещения они ссорятся настолько, что никакие двое из них после этого вместе мороженое не едят. К концу года выяснилось, что в дальнейшем они могут ходить в кафе-мороженое только поодиночке. Докажите, что если число посещений было к этому времени больше 1, то оно не меньше числа учащихся в школе.
\end{task}

% 7. Вечерние визиты класса
\begin{task}[3]
30 учеников одного класса решили побывать друг у друга в гостях. Известно, что ученик за вечер может сделать несколько посещений, и что в тот вечер, когда к нему кто-нибудь должен прийти, он сам никуда не уходит. Покажите, что для того, чтобы все побывали в гостях у всех,
\begin{enumerate}
    \item[а)] четырёх вечеров недостаточно,
    \item[б)] пяти вечеров также недостаточно,
    \item[в)] а десяти вечеров достаточно,
    \item[г)] и даже семи вечеров тоже достаточно.
\end{enumerate}
\end{task}

% 8. Выборы мэра и знакомые
\begin{task}[3]
У каждого из жителей города $N$ число знакомых составляет не менее 30\% населения города. Житель идёт на выборы, если баллотируется хотя бы один из его знакомых. Докажите, что можно так провести выборы мэра города $N$ из двух кандидатов, что в них примет участие не менее половины жителей.
\end{task}

% 9. Комитеты в Думе
\begin{task}[3]
В Думе 1600 депутатов образовали 16000 комитетов по 80 человек в каждом. Докажите, что найдутся два комитета, имеющие не менее четырёх общих членов.
\end{task}

\vspace{1.5em}
\noindent\textbf{Примечание.}
\begin{quote}
Напоминание: задачи, имеющие сложность 1 должны уметь решать все. На решение этих задач даётся дедлайн – две недели (на первый раз 09.10.2025).
\end{quote}

\end{document}
