\documentclass{article}
\usepackage{fontspec}
\usepackage{polyglossia}
\setdefaultlanguage{russian}
\setmainfont{Times New Roman}

\usepackage{enumitem}

\title{diskret}
\author{Артём Басыров}
\date{October 2025}

\begin{document}

\maketitle

\section{task 1a}
$$
A \subseteq (B \land C) \iff \exists x \in A : x \in (B \land C) \iff x \in B \land x \in C \iff A \subseteq B \land A \subseteq C
$$
\section{task 1b}
$$
A \subseteq (B / C) \iff \exists x \in A : x \in (B / C) \iff x \in B \land x \notin C \iff A \subseteq B \land A \not\subseteq C
$$
\section{task 2a}
$$
P(A \cap B) = \{ X \mid X \subseteq A \cap B \} = \{ X \mid X \subseteq A \land X \subseteq B \} = P(A) \cap P(B)
$$

\section{task 2б}
$$
X \in P(A) \cup P(B) \implies X \subseteq A \text{ или } X \subseteq B \implies X \subseteq A \cup B
$$
следовательно,
$$
P(A) \cup P(B) \subseteq P(A \cup B)
$$

\section{task 2в}
$$
X \in P(A \setminus B) \implies X \subseteq A \setminus B
$$
Если $X \neq \emptyset$, то
$$
X \subseteq A,\quad X \cap B = \emptyset \implies X \notin P(B)
$$
и
$$
X \in P(A) \setminus P(B)
$$
Если $X = \emptyset$, то $\emptyset \in \{\emptyset\}$.

$$
P(A \setminus B) \subseteq (P(A) \setminus P(B)) \cup \{\emptyset\}
$$

\section{task 4}
\begin{enumerate}[label=\arabic*)]
  \item Область: $\mathbb{R}$; инъективна, сюръективна, биекция.
  \item Область: $[1, +\infty)$; не инъективна, не сюръективна.
  \item Область: $\mathbb{R}$; биекция.
  \item Область: $(0, +\infty)$; инъективна, не сюръективна.
  \item Область: $[-1, 1]$; инъективна и сюръективна — биекция.
  \item Область: $[0, 1]$; инъективна и сюръективна — биекция.
  \item Область: $[-1, 1]$; не инъективна, сюръективна.
  \item Область: всё $\mathbb{R}$; не инъективна, не является биекцией.
\end{enumerate}

\section{task 5}
пусть $x \in A$, $y \in B$, $z \in C$ \\
а) нет, т.к. f не гарантирует инъективность \\
б) да, т.к. f сюръективна => покрывает всё множество C \\
в) да, т.к. все $x \in A$ имеют ! пару y $\in$ B и все y$\in$B имеют ! пару z $\in$ C \\
г) нет, т.к. не у всех y есть прообраз x\\
д) да, т.к. все x являются прообразами \\
е) да, т.к. все значения z имеют прообраз \\

\end{document}
